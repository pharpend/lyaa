\chapter{Pedantry}

There are a number of points I would like to get in, but they don't really fit
in the rest of the book. So, here they are.

\section{$\l$-calculus}

Perhaps the laziest group of mathematicians are those who study
$\l$-calculus. In fact, one of the things they study is how to read their own
notation in the laziest manner possible! They don't even try to hide it!

$\l$ is the Greek letter ``lambda.'' It's pronounced ``lamb-duh.'' ``Lamb''
being the animal, and ``duh'' being the thing you say when you are reading
math.

I won't give a thorough introduction to $\l$-calculus. I will, however, use some
of their ideas throughout the book, so I want to explain them here.

There are, as far as we are concerned, four things that you do in $\l$-calculus.

\begin{enumerate}
\item The first thing is a $\l$-abstraction. Basically, it's a way to write a
  function without giving it a name. It's a convenient notation convention that
  I'll use.
  
  Take this function, which operates on $\Z$:
  
  \[ f(x) = 2 \times x \]

  If we wanted to use it without giving it a name, we could write it as

  \[ \l x \to 2 \times x \]
  
\item The next is an $\n$-reduction. It's a way to write functions with as few
  free variables as possible.

  So, let's again consider the function
  
  \[ \l x \to 2 \times x \]
  
  Well, $2 \times$ is obviously a unary operator - it takes some argument, and
  spits out the same argument, multiplied by two.

  So, you could write the previous function as

  \[ \l \to 2 \times \]
  
  By convention, if the $\l$-abstraction takes no free arguments (that is,
  arguments that are given a name), then we just exclude the ``$\l \to$'' part
  of the function.

  So, we would write the above as

  \[ 2 \times \]
  
  Isn't that much clearer?
  
\item The next is $\B$-reduction. Let's look at this function:

  \[ \l x, y \to x \times y\]
  
  Well, what's that function, with the argument $x=5$? 

  \[ \l 5, y \to 5 \times y \]

  Well, since 5 is always there, we can just rewrite the function as

  \[ \l y \to 5 \times y \]

  That's called a $\B$-reduction - where you ``partially apply'' a function, so
  to speak - you non-exhaustively apply arguments to a function, and see what
  the new function looks like. 
\end{enumerate}

I would encourage the reader to try combining these ideas, and see what he can
come up with.

\section{Programming}

\begin{quotation}
  I think everybody in this country should learn how to program a
  computer $\ldots$ because it teaches you how to think.

  -- Steve Jobs
\end{quotation}

\begin{quotation}
  When you're programming, you're teaching, possibly the stupidest thing in the
  entire universe, a computer, how to do something.
  
  -- Gabe Newell
\end{quotation}

You can read as many books as you want. Nothing will teach you the discipline of
logical thought like programming. Programming teaches you how to break down
problems into smaller, discrete problems, and to compose those solutions. If you
intend to learn more about math or physics, or even something blue-collar like
biology or engineering, I strongly recommend learning how to program, because,
as Steve Jobs said, it teaches you how to think.

The people on Python's IRC channel suggested
\href{http://learnpythonthehardway.org/book/}{Learn Python the Hard Way} as a
good starting point, if you don't already know how to program.

My favorite programming language is
\href{http://en.wikipedia.org/wiki/Haskell_(programming_language)}{Haskell}. Haskell
is probably the best programming language when it comes to
mathematically-oriented thinking. If you already know how to program, but don't
know Haskell, I suggest Miran Lipovača's wonderful book,
\href{http://learnyouahaskell.com/chapters}{Learn You a Haskell for Great Good!}

Haskell, the programming language, is named after Haskell Curry, the namesake
behind the notation we used for expressing type signatures.

For complicated concepts and algorithms, I will likely include explanations with
Haskell code. 
