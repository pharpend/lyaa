\chapter{Categorizing things}

I did my best in the first chapter to establish that mathematical notation works
best if it is designed with the end of laziness in mind. So, this chapter is
devoted entirely to laziness. 

Perhaps the laziest group of mathematicians are those who study
$\l$-calculus. In fact, one of the things they study is how to read their own
notation in the laziest manner possible! They don't even try to hide it!

$\l$ is the Greek letter ``lambda.'' It's pronounced ``lamb-duh.'' ``Lamb''
being the animal, and ``duh'' being the thing you say when you are reading
math.

I won't give a thorough introduction to $\l$-calculus. I will, however, use some
of their ideas throughout the book, so I want to explain them here.

There are, as far as we are concerned, four things that you do in $\l$-calculus.

\begin{enumerate}
\item The first thing is a $\l$-abstraction. Basically, it's a way to write a
  function without giving it a name. It's a convenient notation convention that
  I'll use.
  
  Take this function, which operates on $\Z$:
  
  \[ f(x) = 2 \times x \]

  If we wanted to use it without giving it a name, we could write it as

  \[ \l x \to 2 \times x \]
  
\item The next is an $\n$-reduction. It's a way to write functions with as few
  free variables as possible.

  So, let's again consider the function
  
  \[ \l x \to 2 \times x \]
  
  Well, $2 \times$ is obviously a unary operator - it takes some argument, and
  spits out the same argument, multiplied by two.

  So, you could write the previous function as

  \[ \l \to 2 \times \]
  
  By convention, if the $\l$-abstraction takes no free arguments (that is,
  arguments that are given a name), then we just exclude the ``$\l \to$'' part
  of the function.

  So, we would write the above as

  \[ 2 \times \]
  
  Isn't that much clearer?
  
\item The next is $\B$-reduction.
\end{enumerate}
